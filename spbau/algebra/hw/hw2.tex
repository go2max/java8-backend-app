\documentclass[a4paper,12pt]{article}

\usepackage[T2A]{fontenc} 
\usepackage[utf8]{inputenc}
\usepackage[english,russian]{babel}
\usepackage{listings}
\usepackage[dvips]{graphicx}
\usepackage{indentfirst}
\usepackage{color}
\usepackage{hyperref}
\usepackage{amsmath}
\usepackage{amssymb}
\usepackage{geometry}
\geometry{left=2cm}
\geometry{right=2cm}
\geometry{top=2cm}
\geometry{bottom=2cm}

\graphicspath{{images/}}

\begin{document}
\sloppy

\lstset{
	basicstyle=\small,
	stringstyle=\ttfamily,
	showstringspaces=false,
	columns=fixed,
	breaklines=true,
	numbers=right,
	numberstyle=\tiny
}

\newtheorem{Def}{Определение}[section]
\newtheorem{Th}{Теорема}
\newtheorem{Lem}[Th]{Лемма}
\newenvironment{Proof}
	{\par\noindent{\bf Доказательство.}}
	{\hfill$\scriptstyle\blacksquare$}
\newenvironment{Solution}
	{\par\noindent{\bf Решение.}}
	{\hfill$\scriptstyle\blacksquare$}


\begin{flushright}
	Кринкин М. Ю. группа 504 (SE)
\end{flushright}

\section{Домашнее задание 2}

\paragraph{1.1} Пусть $F,H \le G$. Докажите, что $F \cup H \le G$, если и только если $F \le H$ или $H \le F$.
\begin{Proof}
Покажем, что $F \cup H \le G \Rightarrow \left[\left(F \le H\right) \lor \left(H \le F\right)\right]$

Предположим обратное, т. е. $\exists x \in H \setminus F$ и $\exists y \in F \setminus H$, тогда можно сказать, что $x^{-1} \in H \setminus F$, т. к.  противном случае он принадлежал бы $H \cap F$ и значит принадлежал бы $F$, что необходимо влечет $x \in F$, аналогичное утверждение справделиво и для $y$, т. е. $y^{-1} \in F \setminus H$. Рассмотрим элемент $z = xy \in H \cap F$, тогда можно сказать, что $y = x^{-1} z$ и $x = z y^{-1}$, но это значит, что $y \in H$ и $x \in F$, что не возможно, следовательно либо $F \le H$ либо $H \le F$.

Покажем теперь, что $\left[\left(F \le H\right) \lor \left(H \le F\right)\right] \Rightarrow F \cup H \le G$.

Это утерждение очевидно
\[
	\begin{split}
		& \left|F\right| < \left|H\right| \Rightarrow F \cup H = H \Rightarrow F \cup H \le G \\
		& \left|F\right| > \left|H\right| \Rightarrow F \cup H = F \Rightarrow F \cup H \le G \\
		& \left|F\right| = \left|H\right| \Rightarrow F \cup H = F = H \Rightarrow F \cup H \le G
	\end{split}
\]
\end{Proof}

\paragraph{1.2} Пусть $H \le G$. Докажите, что $\left(G \setminus H\right) \cup \{1\} \le G$, если и только если $H = \{1\}$ или $H = G$.

\begin{Proof}
Покажем, что $\left(H=\{1\}\right) \lor \left(H = G\right) \Rightarrow \left(G \setminus H\right) \cup \{1\} \le G$.

\[
	\begin{split}
		& H=\{1\} \Rightarrow \left(G \setminus H\right) \cup \{1\} = G \\
		& H=G \Rightarrow \left(G \setminus H\right) \cup \{1\} = \{1\}
	\end{split}
\]

Покажем теперь, что $\left(G \setminus H\right) \cup \{1\} \le G \Rightarrow \left[\left(H=\{1\}\right)\cup\left(H=G\right)\right]$

Обозначим $F = \left(G \setminus H\right) \cup \{1\}$. Положим теперь, что $F,H \le G$, тогда по предыдущей задаче из того, что $G = F \cup H$ - группа, следует, что $F \le H$ или $H \le F$, т. е. либо $H = G$ либо $H = \{1\}$
\end{Proof}

\paragraph{1.3} Докажите, что для любого $n \in \mathbb{N}$ в группе $\mathbb{C}^*$ имеется ровно одна подгруппа порядка $n$ (а именно, группа $\mu_n$).

\begin{Proof}
Очевидно, что для любого натурального $n$ существует группа $\mu_n \in \mathbb{C}^*$ корней $n$-ой степени из единицы, покажем теперь, что эта группа единстенна.

Предположим обратное и эта группа не единстенна, т. е. сущестует группа $H \le \mathbb{C}^*$, причем $\left|H\right| = n$ и $\mu_n \not= H$. Все элементы этой группы принадлежат единичному кольцу, покажем это. Пусть это не так и существует элемент $g \in H$ причем $\left|g\right| \not= 1$, тогда вместе с ним  $H$ включены и все различные степени $g$, а их бесконечное число, следовательно и порядок группы бесконечен.

Далее, у группы $H$ сущестует система образующих элементов, т. е. $<g_1, g_2, ... ,g_k> = H$. Т. к. $\left|H\right| = n$ - конечное число, то $\forall g_i \exists t \in \mathbb{N} : {g_i}^{t}=1$, т. е. каждый из элементо порождает некоторую группу корней из едницы, но из того, что их объединение является подгруппой, согласно первой задче вытекает, что среди них есть подгруппа мощности $n$ включающая все остальные в качестве подгрупп, т. е. $H = \mu_n$
\end{Proof}

\paragraph{1.4} Пусть $F,H \le G$, $\left|F\right|,\left|H\right| < \infty$ И $\gcd{\left|F\right|}{\left|H\right|} = 1$. Докажите, что $F \cap H = \{1\}$

\begin{Proof}
Обозначим $I = F \cap H \le F$, тогда $\left(I \le F\right) \land \left(I \le H\right)$, и тогда $\left|I\right|$ делит как $\left|F\right|$ так и $\left|H\right|$, т. е. является их общим делителем и не может превышать 1, единственная группа из одного элемента это $\{1\}$.
\end{Proof}

\paragraph{1.5} Докажите, что любая бесконечная группа имеет бесконечно много подгрупп.

\begin{Proof}
Будем последовательно образовывать циклически подгруппы от элементов группы, если получилась конечная подгруппа, то выбираем элемент исходной группы не лежащий ни в одной полученной конечной подгруппе и образовываем циклическую подгруппу от него. Если все подгруппы конечны, то их число бесконечно и следовательно для такого случая задача доказана. Теперь пусть у нас получилась бесконечная циклическая подгруппа исходной группы имеющая вид:
\[
	..., g^{-3}, g^{-2}, g^{-1}, 1, g, g^2, g^3, ...
\]
выберем в ней элемнт отличный от $1, g, g^{-1}$ и образуем от него циклическую подгруппу, эта подгруппа также бесконечная и не совпадает с исходной группой, для нее проделаем те же самые действия и снова получим подгруппу, так как каждый раз получаем бесконечную подгруппу то продолжать так мы можем бесконечно, следовательно и число подгрупп бесконечно.
\end{Proof}

\paragraph{1.6} Докажите, что любое непустое мультипликатино замкнутое подмножество конечной группы является подгруппой. Приедите пример бесконечной группы, для которой это утерждение не верно.

\begin{Solution}
$\forall g \in G : ord\left(g\right) < \infty$, $G$ - группа. Следовательно $\forall g \in H \subset G \exists n \in \mathbb{Z}_{\ge 0} : g^n = 1 $, кроме того $\forall g \in H : \left[g^n = 1 \Rightarrow g^{n-1} = g^{-1}\right]$, таким образом $H \le G$.

Примером бесконечной группы, для которой это не верно может служить группа $\left<\mathbb{Z}, +\right>$, подмножество $\mathbb{Z}_{\ge 0}$ замкнуто относительно групповой операции в $\left<\mathbb{Z}, +\right>$, но очевидно не является группой.
\end{Solution}

\paragraph{1.7} Пусть $|G| \in 2\mathbb{N}$, $H < G$ и $|H| = |G|/2$. Докажите, что $\forall g \in G \left(g^2 \in H\right)$

\begin{Solution}
Для элементов $g \in H$ очевидно выполняется условие $g^2 \in H$. Рассмотрим элементы не лежащие в $H$. Все множество распадается на два класса смежности $H, gH$, где $g \in G \setminus H$. Множество $g^2H$ совпадает с одним из этих классов, т. е. либо $g^2H = H$ либо $g^2H = gH \Leftrightarrow gH = H$ второе очевидно противоречит условию, что $H \not= gH$
\end{Solution}

\paragraph{1.9} Докажите, что $\mathbb{R}/\mathbb{Z} \cong \mathbb{T}$

\begin{Solution}
Пусть $x \in \mathbb{Z}$, тогда $e^{2 \pi x} = 1$. Это гомоморфизм:
\begin{enumerate}
\item $x+y \mapsto e^{2 \pi x}e^{2 \pi y} = e^{2 \pi \left(x+y\right)}$

\item $-x \mapsto e^{-2 \pi x}, e^{-2 \pi x} e^{2 \pi x} = e^0 = 1$

\item $0 \mapsto e^0 = 1$
\end{enumerate}
Ядром гомоморфизма являются целые числа, так как $e^{2 \pi x \in \mathbb{N}} = e^0 = 1$, следовательно группы изоморфны
\end{Solution}

\paragraph{1.10} Докажите, что $\mathbb{Q}/\mathbb{Z} \cong \mu_{\infty}$

\begin{Solution}
Рассмотрим гомоморфизм $x \in \mathbb{Q} \mapsto e^{2 \pi x}$ (доказательство того, что он гомоморфизм аналогично предыдущей задаче).
Ядром гомоморфизма являются целые числа, следовательно группы изоморфны.
\end{Solution}

\paragraph{1.11} Докажите, что ${\mathbb{C}*}/{\mathbb{R}_{>0}} \cong \mathbb{T}$.

\begin{Solution}
Рассмотрим отображение: $\mathbb{C}* \rightarrow \mathbb{T}_{>0}$, в котором $re^{ix} \mapsto e^{ix}$, оно является гомоморфизмом:
\begin{enumerate}
\item $r_1 r_2 e^{i \left(x_1 + x_2\right)} = r_1 e^{i x_1} r_2 e^{i x_2} \mapsto e^{i \left(x_1 + x_2\right)}$

\item $\frac{1}{r} e^{i \left(-x\right)} \mapsto e^{i \left(-x\right)}, e^{- i x} e^{i x} = e^0 = 1$

\item $1 = e^{i 0} \mapsto e^{i 0} = 1$
\end{enumerate}
Ядром гомоморфизма являются положетильные вещественные числа, которые очевидно переводятся в 1, следовательно группы изоморфны.
\end{Solution}

\paragraph{1.12} Докажите, что ${\mathbb{C}*}/\mathbb{T} \cong \mathbb{R}_{>0}$

\begin{Solution}
Рассмотрим отображение, которое $r e^{i x} \mapsto r$. Это отображение является гомоморфизмом:
\begin{enumerate}
\item $r_1 e^{i x_1} r_2 e^{i x_2} \mapsto r_1 r_2$

\item $\frac{1}{r} e^{- i x} \mapsto \frac{1}{r}$

\item $1 e^{i 0} \mapsto 1$
\end{enumerate}
Ядром гомоморфизма, являются все элементы единицной окружности комплексной плоскости, следовательно группы изоморфны.
\end{Solution}

\paragraph{1.14} Пусть $n \in \mathbb{N}$. Докажите, что ${\mathbb{C}*}/{\mu_{n}} \cong \mathbb{C}*$

\begin{Solution}
Рассматриваем отображение $x \in \mathbb{C}* \mapsto x^n$. Для кажлго числа $r e^{i x} \in \mathbb{C}*$ существует корень $n$-ой степени $r^{1/n} e^{i \frac{x}{n}}$, следовательно образ отображения совпадает с $\mathbb{C}*$. Покажем, что отображение является гомоморфизмом:
\begin{enumerate}
\item $r_1 e^{ix_1} r_2 e^{ix_2} \mapsto r_1^n e^{i n x_1} r_2^n e^{i n x_2} = \left(r_1r_2\right)^n e^{i n \left(x_1 + x_2\right)}$

\item $\frac{1}{r} e^{-ix} \mapsto {\frac{1}{r}}^n e^{- i n x}, r^n {\frac{1}{r}}^n e^{i n \left(x - x\right)} = 1$

\item $1 \mapsto 1^n = 1$
\end{enumerate}
Ядром гомоморфизма являются все корни $n$ -ой степени из единицы, следовательно группы изоморфны.
\end{Solution}

\paragraph{2.1} Докажите, что для любых $g, h \in G$ имеют место следующие тождества: $ord(g) = ord(g^{-1})$, $ord(g) = ord(hgh^{-1})$ и $ord(gh) = ord(hg)$

\begin{Solution}
Пусть поярдки $g$ и $g^{-1}$ отличаются, причем порядок $g = m < ord(g^{-1})$, тогда $1 = g^mg^{-m} = g^{-m} \not= 1$ - получили противоречие, следовательно порядки равны (в силу того, что $(g^{-1})^{-1} = g$).

Пусть $ord(g) = m < \infty$, тогда $(hgh^{-1})^m = hgh^{-1}hgh^{-1}hgh{^-1}... hgh^{-1} = h g^m h^{-1} = h h^{-1} = 1$. Тем самым показано, что порядок $g$ кратен порядку $hgh^{-1}$, покажем теперь, что порядок $hgh^{-1}$ не меньше порядка $g$. Предположим, что $ord(hgh^{-1}) = n < m$, тогда $1 = (hgh^{-1})^n = hgh^{-1}hgh^{-1}hgh^{-1}...hgh^{-1} = hg^nh^{-1} \Rightarrow h^{-1}h = g^n$, а это невозможно, так как порядок $g$ равен $m$. Из последнего так же следует, что невозможна ситуация, когда порядок сопряженного конечен, а порядок самого элемента бесконечен.

Теперь пусть $ord(gh) = n$, тогда $1 = (gh)^n = ghghgh...gh = g (hg)^{n-1} h \Rightarrow g^{-1}h^{-1} = (hg)^{-1} = (hg)^{n-1} \Rightarrow 1 = (hg)^n$. АНалогичное справедливо и для случая $ord(hg) = n$, следовательно, если $gh$ или $hg$ иемеет конечный порядок, то порядки обоих элементов совпадают. Следовательно, не возможная ситуация, когда один из них имеет конечный порядок, а второй бесконечный, следовательно и для случая бесконечного порядка все доказано.
\end{Solution}

\paragraph{2.2} Пусть $f: G \rightarrow K$ - гомоморфизм групп, $g \in G$ и $ord(g) < \infty$. Докажите, что $ord(f(g)) < \infty$ и $ord(g)$ делится на $ord(f(g))$.

\begin{Solution}
Если $ord(g) = m$, то $1_G = g^m \mapsto f(g)^m = 1_K$ по определению гомоморфизма. Теперь, если $ord(f(g)) = n < m$, то справедливо утверждение, что $\exists k \in \mathbb{Z} : m = kn$, докажем его, пусть это не так, тогда $m = kn + r$ и $r < n$, тогда $f(g)^m = f(g)^{kn}f(g)^r = f(g)^r \not= 1$, так как $n$ наименьше положительное целое число, для которого $f(g)^n = 1$.
\end{Solution}

\paragraph{2.3} Пусть $|G| \in 2\mathbb{N} + 1$. Докажите, что для любого $g \in G$ существует ровно один такой $h \in G$, что $h^2 = g$.

\begin{Solution}
В конечной группе нечетной мощности не может быть подгрупп четной мощности, следовательно порядки всех элементов группы конечны и нечетны. Пусть теперь существуют $h_1 \not = h_2$, причем $h_1^2 = g = h_2^2$. Пусть $ord(h_1) = 2 k_1 + 1$ и $ord(h_2) = 2 k_2 + 1$, тогда $h_1^{(2k_1 + 1)(2k_2 + 1)} = h_2^{(2 k_1 + 1)(2 k2 + 1)} \Rightarrow h_1^{4 k_1 k_2 + 2 k_1 + 2 k_2 + 1} = g^{2k_1k_2 + k_1 + k_2}h_1 = g^{2k_1 k_2 + k_1 + k_2}h_2$ сокращаем слева и получаем, что $h_1 = h_2$. Таким образом если корень существует, то он единственный. Но из этого также следует, что все квадраты элементов группы различны между собой, и их ровно столько, сколько элементов группы, а значит все элемнты группы представимы как квадрат.
\end{Solution}

\paragraph{2.4} Пусть $|G| \in 2\mathbb{N}$. Докажите, что $\exists g \in G (ord(g) = 2)$

\begin{Solution}
Требуется доказать, что существует элемент обратный сам себе и отличный от единицы. Пусть это не так, и ни один элемент кроме единицы не обратен сам себе, но в этом случае количество элементов в группе нечетно (каждый элемент вместе с обратным образует пару, такие пары не пересекаются, так как обратный элемент единственен, плюс единица), получаем противоречие.
\end{Solution}

\paragraph{2.5} Опишите все гомоморфизмы из группы $\mathbb{Z}$ в произвольную группу.

\begin{Solution}
$\mathbb{Z}$ - бесконечная циклическая группа. В качестве образующего элемента в ней могут выступать либо 1, либо -1. Следовательно любой гомоморфизм однозначно определяется тем, что он сопоставляет -1 или 1, при условии что выполняются все другие условия гомоморфизма.
\end{Solution}

\paragraph{2.8} Пусть группа $G$ нетривиальна и все ее элементы имеют одинаковый порядок $p$. Докажите, что $p$ - простое число.

\begin{Solution}
Образуем циклическую подгруппу $H_g$ от $g \in G$, порядок подгруппы равен $p$. Если $p = nk$ - составное число ($n$ и $k$ отличны от 1 и $p$), то $g^{nk} = 1 \Rightarrow (g^n)^k = 1 \Rightarrow h^k = 1$. Следовательно существует элемент порядок, которого отличен от $p$.
\end{Solution}

\paragraph{2.11} Пусть $n \in \mathbb{N}$; отождествим группу $S_{n-1}$ с подгруппой $\{u \in S_n | u(n) = n\}$ группы $S_n$. Опишите множества $S_n/S_{n-1}$ и $S_{n-1} \backslash S_{n}$.

\begin{Solution}
Получается, что $S_{n-1}$ - перестановка не изменяющая $n$. Тогда умножая некоторую перестановку $u \in S_n$ на перестановку слева на $v \in S_{n-1}$ получаем перестановку, в которой $uv(n) = u(n)$, аналогично, если умножаем справа $vu(n) = u(n)$. Т. е. получаем множества перестановок, такие, что перестановки из каждого множества переводят элемент $n$ в один и тот же элемент.
\end{Solution}

\paragraph{2.12} Докажите, что группы $\mathbb{Q}$ и $\mathbb{Q}_{>0}$ не являются изоморфными.

\begin{Solution}
Пусть существует гомоморфизм из $f:\mathbb{Q} \rightarrow \mathbb{Q}_{>0}$. И этот гомоморфизм сопоставляет $\frac{n_1}{d_1} \mapsto \frac{n_2}{d_2}$. Разложим $n_2$ и $d_2$ на произведение простых сомножителей, такое разложение имеет конечное число множителей. Представить $\frac{n_2}{d_2}$ как произведение двух нескоратимых дробей (произведение положительных рациональных чисел) можно конечным числом способов. Представить $\frac{n_1}{d_1}$ в виде суммы двух рациональных чисел (нескоратимых дробей) можно бесконечным числом способов, и для каждого такого представления $\frac{n_1}{d_1} = \frac{n_1'}{d_1'} + \frac{n_1''}{d_1''} \mapsto \frac{n_2'}{d_2'} \frac{n_2''}{d_2''} = \frac{n_2}{d_2}$ Но так как слева возможно бесконечное число вариантов представления, а справа конечное, то какие-то из множеств $\{\frac{n_1'}{d_1'}\}$ и $\{\frac{n_1''}{d_1''}\}$ переходят в одни и те же элементы из множеств $\{\frac{n_2'}{d_2'}\}$ и $\{\frac{n_2''}{d_2''}\}$, а значит если такой гомоморфизм существует, он не является биективным, то есть группы не изоморфны.
\end{Solution}

\paragraph{2.15} Пусть $H \trianglelefteq G$ и все элементы групп $H$ и $G/H$ имеют конечный порядок. Докажите, что все элементы группы $G$ имеют конечный порядок.

\begin{Solution}
Так как $G/H$ - группа классов смежности образует разбиение $G$ на непересекающиеся блоки и каждый элемент этой группы имеет конечный порядок, то $\left|G : H\right| < \infty$. По теореме Лагрнажа $\left|G\right| = \left|H\right| \left|G : H\right|$. Так как в $H$ все эелементы имеют конечный порядок, то и $\left|H\right| < \infty$, в итоге имеем, что $\left|G\right| < \infty$, а в конечной группе не может быть элементов бесконечного порядка.
\end{Solution}

\paragraph{3.1} Пусть $|G| < p^2$, где $p$ - простое число, $F,H < G$ и $|F| = |H| = p$. Докажите, что $F = H$, а также выведите отсюда, что $H \vartriangleleft G$

\begin{Solution}
В случае, если $|G| = p$ все очевидно и такой случай мы рассматривать не будем. $F$ и $H$ - циклические подгруппы простого порядка (так как в них содержит как минимум 1 элемент с порядком выше первого, а так как порядок подгрупп простой, порядок этогих элементов совпадает с порядками подгрупп), по налогичным рассуждениям получим, что если элементы пересекаются хотябы по одному неединичному элементу, то они совпадают. Предположим теперь, что подгруппы пересекаются только по единичному элементу. Рассмотрим произведение $ab$, где $a \in H$ и $b \in F$ и $a \not= 1$ и $b \not= 1$, любое такое произведение не лежит ни в $H$ ни в $F$, в противном случае и $a$ и $b$ будут лежать в одной подгруппе, что противоречит исходному предположению. Пусть теперь все произведения вида $ab$ различны, тогда их число $(p-1)^2 = p^2 - 2p + 1$, но в объединении $F$ и $H$ содержится $2p-1$ элементов, что вместе дает $p^2$, что невозможно. Теперь, пусть $a_1b_1 = a_2b_2$ для некоторых $a_1,a_2 \in H$ и $b_1,b_2 \in F$, но тогда $a_2^{-1}a_1 = b_2b_1^{-1}$ и так как $a_1, a_2, b_1, b_2 \not= 1$, то такое возможно в случае если $F$ и $H$ пересекаются по неединичному элементу, или в случае, если $a_2 = a_1$ и $b_2 = b_1$, что опять же противочречит предположению, таким образом все группы порядка $p$ совпадают.

То что такая подгруппа является нормальным делителем можно показать следующим образом, рассмотрим подгруппу $gHg^{-1}$ для любого $g \in G$ это группа мощностью $p$, а значит по доказанному выше совпадает с исходной, таким образом, группа с каждым элементов включает и все сопряженные, следовательно она является нормальным делителем.
\end{Solution}

\paragraph{3.2} Пусть $F,G \le G$ и $|G:F|, |G:H| < \infty$. Докажите, что $|G:\left(F\cap H\right)| \le |G:F||G:H|$

\begin{Solution}
Рассмотрим множество левых смежных классов по $F \cap H$. Элементы $a$ и $b$ лежат в одном смежном классе по пересечению, в случае, если $a^{-1}b \in F \cap H$, но это значит, что $a^{-1}b \in H$ и $a^{-1}b \in F$, т. е. все левосторонние смежные классы по пересечению, являются пересечением левосторонних смежных классов по подгуппам $F$ и $H$. И количество таких пересечений не может быть больше чем произведение индексов $F$ и $H$ в $G$
\end{Solution}

\paragraph{3.3} Пусть группа $G$ абелева, $H \le G$ и множество $\{g \in G | ord(g) = \infty\}$ непусто и содержтся в подгруппе $H$. Докажиет, что $H=G$.

\begin{Solution}
Пусть $h \in H$ - элемент бесконечного порядка, а $g \in G$ - элемент конечноного порядка, они, очевидно, не взаимообратны. Тогда олучается, что $gh \in H$, так как он тоже является элементом конечного порядка, докажем это, предположив обратное, т. е. $\exists n > 0 \in \mathbb{N} : (gh)^n = 1$, тогда из коммутативности следует, что $g^n h^n = 1$, но $g^n \not= 1$ и $g^n \not= h^{-n}$, следовательно приложение не верно. Таким образом $gh \in H$, но это влечет, что $ g = gh h^{-1} \in H$, т. е. со всеми элементами бесконечного порядка в $H$ содержатся и все элементы конечного порядка, следовательно группа и подгруппа совпадают.
\end{Solution}

\paragraph{3.5} Пусть $|G| < \infty$ и $H < G$. Докажите, что $\cup_{g \in G} gHg^{-1} \not= G$.

\begin{Solution}
Выглядит, вроде просто, но пока не дается
\end{Solution}

\paragraph{3.8} Пусть $g \in G$. Докажите, что $\{x \in G | xg = gx\} \le \{x \in G | xg = gx \text{или} xg = g^{-1}x\} \le G$

\begin{Solution}
Покажем, что $\{x \in G | xg = gx\}$ - подгруппа в $G$. Ассоциативность очевидна, так как все $x \in G$, существование единицы очевидна, так как единица коммутирует со всеми элементами, следовательно принадлежит множеству. Докажем, что обратный элемент так же лежит в множестве $x^{-1}x g = g x^{-1}x \Rightarrow x^{-1} g x = g x^{-1} x \Rightarrow x^{-1} g = g x^{-1}$. Далее покажем закнутость относительно групповой операции $xyg = xgy = gxy$.

Далее покажем, что если $xg = gx$, тогда $xg^{-1} = g^{-1}x$. $(gx)^{-1} = (xg)^{-1} \Rightarrow x^{-1}g^{-1} = g^{-1}x^{-1}$, а так как элементы с таким свойством образуют группу, то и $x g^{-1} = g^{-1} x$.

Теперь покажем, что $\{x \in G | xg = gx \text{или} xg=g^{-1}x\}$. Ассоциативность и наличие единицы опять же очевидны. Покажем, что обратный к элементу, для которого $xg = g^{-1}x$ опять же лежит в множестве $(xg)^{-1} = (g^{-1}x)^{-1} \Rightarrow g^{-1}x^{-1} = x^{-1}g$. Теперь покажем замкнутость относительно групповой операции. Пусть $xg = g^{-1}x$ и $yg = g^{-1}y$, тогда их произведение $xyg = xg^{-1}y = gxgg^{-1}y = gxy$, т. е. произведение таких элементов коммутирует с $g$, значит принадлежит множеству. Теперь пусть $xg = gx$ и $yg = g^{-1}y$, рассмотрим их первое произведение $xyg = x g^{-1} y = g^{-1} xy$, т. е. опять же принадлежит множеству, теперь $yxg = ygx = g^{-1}yx$, т. е множество замкнуто относительно групповой операции, то есть оба множества являются подгруппами в $G$ и очевидно, что первое является подгруппой второго.
\end{Solution}
\end{document}
